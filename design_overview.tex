\documentclass[11pt, a4paper, fleqn]{article}

%%%%%%%%%%%%%%%%%%%%%%%%%%%%%%%%%%%%%%%%%%%%%%%%%%%
%%%                   BEGIN BS                  %%%
%%%%%%%%%%%%%%%%%%%%%%%%%%%%%%%%%%%%%%%%%%%%%%%%%%%
\usepackage{color}
\usepackage{parskip}
\usepackage{listings}
\usepackage{scrextend}
\definecolor{dkgreen}{rgb}{0,0.6,0}
\definecolor{gray}{rgb}{0.5,0.5,0.5}
\definecolor{mauve}{rgb}{0.58,0,0.82}
\definecolor{class}{rgb}{0,.25,.53}
\definecolor{method}{rgb}{.1,.46,.99}

\lstset{ %
language=Java,                % choose the language of the code
basicstyle=\footnotesize\ttfamily,       % the size of the fonts that are used for the code
numbers=left,                   % where to put the line-numbers
numberstyle=\tiny\color{gray},      % the size of the fonts that are used for the line-numbers
stepnumber=1,    
firstnumber=1,
numberfirstline=true                   % the step between two line-numbers. If it is 1 each line will be numbered
numbersep=5pt,                  % how far the line-numbers are from the code
backgroundcolor=\color{white},  % choose the background color. You must add \usepackage{color}
showspaces=false,               % show spaces adding particular underscores
showstringspaces=false,         % underline spaces within strings
showtabs=false,                 % show tabs within strings adding particular underscores
tabsize=2,          % sets default tabsize to 2 spaces
captionpos=b,           % sets the caption-position to bottom
breaklines=true,        % sets automatic line breaking
breakatwhitespace=false,    % sets if automatic breaks should only happen at whitespace
 keywordstyle=\color{blue},          % keyword style
  commentstyle=\color{dkgreen},       % comment style
  stringstyle=\color{mauve},         % string literal style
  morekeywords={switch, case}
}

%\usepackage[demo]{graphicx}
\usepackage[T1]{fontenc}
\usepackage[bitstream-charter]{mathdesign}
\renewcommand{\ttdefault}{pcr}
\renewcommand{\sfdefault}{\familydefault}
\newcommand{\class}[1]{{\color{class}\textbf{\lstinline{#1}}}}
\newcommand{\method}[3]{%
\lstinline{#1\ }%
{\color{method}\textit{\lstinline{#2}}}%
\lstinline{(#3)}%
}
\newcommand{\java}[1]{\lstinline{#1}}

\newcommand{\pic}[3]{
\begin{figure}[htb]
\centering
\includegraphics[scale=.6]{#1}
\caption{#3}
\label{fig-#2}
\end{figure}
}
\usepackage{hyperref}
%%%%%%%%%%%%%%%%%%%%%%%%%%%%%%%%%%%%%%%%%%%%%%%%%%%
%%%                   EIND BS                   %%%
%%%%%%%%%%%%%%%%%%%%%%%%%%%%%%%%%%%%%%%%%%%%%%%%%%%

\title{Design and Style Guide}
\author{Jouke van der Maas \& Koen Keune}

\begin{document}
\maketitle

\section{Overview}
During this project, a working sudoku solver will be implemented. Only
logical solving strategies will be used; there will be no brute-force methods.
The underlying principle to the method described in this document is that once
possible numbers have been calculated for each square, they can only be removed.
By doing this repeatedly, one of two states can always be achieved. From these
states, a new number can be entered and the process can repeat.

\subsection{End-states}
The first end-state from which a new number can be entered is the 
\emph{Single Possibility rule}. This rule states that if a given cell has only one
possible number, that number should be entered. The second end-state is the \emph{%
Only Square rule}. This rule states that if a given number can be entered in only one
cell within a container (row, column or square), that number should be entered. All
other strategies implemented for the project only remove possibilities from cells, leading
to one of these two end-states.

\section{Implementation}
\subsection{Important types}
The class \class{Sudoku} represents the puzzle. It holds \class{CellContainer}s that hold
the individual \class{Cell}s. A \class{CellContainer} can be either a row, column or square
within the sudoku. These can be represented using only one type, because their behaviour is
exactly the same. The \class{Cell} class represents cells within the sudoku. This class can
be in exactly one of two states; it can have a value or it can have a list of possible values.
These two states were not separated in two subclasses of an abstract class for simplicities sake.

The \class{Sudoku} is used by the class \class{Solver}. This class takes a \class{Sudoku} and solves it.
It uses implementations of the interface \class{Strategy}. This interface defines a strategy for
removing possibilities from cells, leading to one of the two end-states.

\subsection{Important methods}
\subsubsection{Sudoku}
\begin{itemize}
\item \method{}{Sudoku}{int[][]}\\
The constructor takes a puzzle represented as a jagged array of ints (\java{int[][]}).

\item \method{CellContainer[]}{getRows}{}\\
      \method{CellContainer[]}{getColumns}{}\\
      \method{CellContainer[]}{getSquares}{}\\
      \method{CellContainer[]}{getAllContainers}{}\\
These methods return the rows, columns and/or squares ordered from left to right, top to bottom.
The return values are arrays because the number of containers never changes.
\end{itemize}
\subsubsection{CellContainer}
\begin{itemize}
\item \method{List<Integer>}{getValues}{}\\
Returns a list of all values that have already been entered in this
\\\class{CellContainer}.
\item \method{Cell[]}{getCells}{}\\
Returns an array containing the cells in this container. This is an array because the number
of cells never changes.
\end{itemize}
\subsubsection{Cell}
\begin{itemize}
\item \method{List<Integer>}{getPossibilities}{}\\
Returns a list containing all possibilities for this cell. Throws a \class{SudokuException} if
the cell has a value.
\item \method{boolean}{hasValue}{}\\
Returns \java{true} if this cell has a value, false otherwise.
\item \method{boolean}{removePossibility}{int}\\
      \method{boolean}{removePossibilities}{Collection<Integer>}\\
These methods remove one or more possible numbers from the list of possibilities. Throws a
\class{SudokuException} if the cell has a value. Returns \java{true} if the possibilities were removed,
\java{false} otherwise.
\item \method{int}{getValue}{}\\
Returns the value if there is one. Throws a \class{SudokuException} if there is not.
\item \method{void}{setValue}{int}\\
Changes the state of the cell from not having a value to having a value. Throws a \class{SudokuException}
if the cell already had a value.
\end{itemize}
\subsubsection{Strategy}
\begin{itemize}
\item \method{boolean}{removePossibilities}{Sudoku}\\
Removes one or more possibilities from the cells in the provided sudoku. Returns \java{true} if
possibilities were removed, \java{false} otherwise.
\end{itemize}
\subsubsection{Solver}
\begin{itemize}
\item \method{void}{solve}{}\\
Solves the sudoku provided in the constructor. The method loops over an array of \class{Strategy}
implementations, calling each one on the sudoku. If a possibility was removed, it tries to fill
in a number and starts again from the top of the list. The list is ordered so cheaper (in terms of
time) strategies are at the top. This means that more complex strategies are only used if they are
necessary. The method returns when the sudoku is solved or when no more possibilities could be removed
and no more numbers could be filled in.
\end{itemize}

\section{Style Guide}
The \href{http://www.oracle.com/technetwork/java/javase/documentation/codeconvtoc-136057.html}{Oracle Code Conventions for Java}
are used as a guideline in naming of methods and types, commenting styles and code layout.
\end{document}
