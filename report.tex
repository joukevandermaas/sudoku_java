\documentclass[11pt, a4paper, fleqn]{article}

%%%%%%%%%%%%%%%%%%%%%%%%%%%%%%%%%%%%%%%%%%%%%%%%%%%
%%%                   BEGIN BS                  %%%
%%%%%%%%%%%%%%%%%%%%%%%%%%%%%%%%%%%%%%%%%%%%%%%%%%%
\usepackage{color}
\usepackage{parskip}
\usepackage{listings}
\usepackage{scrextend}
\definecolor{dkgreen}{rgb}{0,0.6,0}
\definecolor{gray}{rgb}{0.5,0.5,0.5}
\definecolor{mauve}{rgb}{0.58,0,0.82}
\definecolor{class}{rgb}{0,.25,.53}
\definecolor{method}{rgb}{.1,.46,.99}

\lstset{ %
language=Java,                % choose the language of the code
basicstyle=\footnotesize\ttfamily,       % the size of the fonts that are used for the code
numbers=left,                   % where to put the line-numbers
numberstyle=\tiny\color{gray},      % the size of the fonts that are used for the line-numbers
stepnumber=1,    
firstnumber=1,
numberfirstline=true                   % the step between two line-numbers. If it is 1 each line will be numbered
numbersep=5pt,                  % how far the line-numbers are from the code
backgroundcolor=\color{white},  % choose the background color. You must add \usepackage{color}
showspaces=false,               % show spaces adding particular underscores
showstringspaces=false,         % underline spaces within strings
showtabs=false,                 % show tabs within strings adding particular underscores
tabsize=2,          % sets default tabsize to 2 spaces
captionpos=b,           % sets the caption-position to bottom
breaklines=true,        % sets automatic line breaking
breakatwhitespace=false,    % sets if automatic breaks should only happen at whitespace
 keywordstyle=\color{blue},          % keyword style
  commentstyle=\color{dkgreen},       % comment style
  stringstyle=\color{mauve},         % string literal style
  morekeywords={switch, case}
}

%\usepackage[demo]{graphicx}
\usepackage[T1]{fontenc}
\usepackage[bitstream-charter]{mathdesign}
\renewcommand{\ttdefault}{pcr}
\renewcommand{\sfdefault}{\familydefault}
\newcommand{\class}[1]{{\color{class}\textbf{\lstinline{#1}}}}
\newcommand{\method}[3]{%
\lstinline{#1\ }%
{\color{method}\textit{\lstinline{#2}}}%
\lstinline{(#3)}%
}
\newcommand{\java}[1]{\lstinline{#1}}

\newcommand{\pic}[3]{
\begin{figure}[htb]
\centering
\includegraphics[scale=.6]{#1}
\caption{#3}
\label{fig-#2}
\end{figure}
}
\usepackage{hyperref}
%%%%%%%%%%%%%%%%%%%%%%%%%%%%%%%%%%%%%%%%%%%%%%%%%%%
%%%                   EIND BS                   %%%
%%%%%%%%%%%%%%%%%%%%%%%%%%%%%%%%%%%%%%%%%%%%%%%%%%%

\title{Implementing a logical sudoku solver}
\author{Jouke van der Maas \& Koen Keune}

\begin{document}
\maketitle

\section{Overview}
During this project, a sudoku solver was implemented. Only
logical solving strategies were used; there are no brute-force methods.
The underlying principle to the solve method is that once
possible numbers have been calculated for each square, they can only be removed.
Removal of possible numbers is achieved by a strategy.
By trying strategies repeatedly, one of two states can always be achieved. From these
states, a new number can be entered and the process can repeat.

\subsection{Terminology}
A sudoku consists of nine \emph{rows}, \emph{columns} and \emph{squares}, called containers. Each of these
contains nine cells. Cells can have a value (numbers one through nine) or a set of up to nine possibilities.

\subsection{End-states}
The first end-state from which a new number can be entered is the 
\emph{Single Possibility rule}. This rule states that if a given cell has only one
possible number, that number should be entered. The second end-state is the \emph{%
Only Square rule}. This rule states that if a given number can be entered in only one
cell within a container (row, column or square), that number should be entered. All
other strategies implemented for the project only remove possibilities from cells, leading
to one of these two end-states.

\section{Implementation}
The implementation is based around the use of strategies, implementations of the interface
\class{Strategy}. The solver uses these to reduce the sudoku to one of the two end-states.
The solver, sudoku representation and UI also support sudokus of other sizes (i.e. a 6x6
sudoku), but no support was added for loading these from a file.

\subsection{Used Strategies}
Following are the strategies that are used in this particular order based on complexity.
\subsubsection{Basic Strategy}
The most basic solving strategy (implemented in \class{OneOfEachStrategy}). Simply removes
values that have already been entered for each row, column and square.
\subsection{Locked Strategy}
Searches for 2 or 3 possibilities in a column that are the same and don't 
occur in another place in the square. If one is found, remove those possibilities on 
every other place on the same column.
\subsection{Double-Locked Strategy}
Searches for possibilities that occurs in one row (or column) in a square
and another row (or column) in another square so that blocks possibilities
in those rows (or columns) in the third square.
\subsection{Naked Group Strategy}
Searches for groups of cells that contain the same possibilities. For example,
a group of three cells that have the same three possibilities ($n$ cells with the same $n$ possibilities). 
When these are found, it is guaranteed the same possibility cannot occur anywhere else within the container,
so those can be removed.
\subsection{Hidden Twin Strategy}
Similar to the naked group. When two possibilities are found in only two cells, and these cells are the same,
nothing else can go in those two cells. This strategy is more complex than the naked group, so only twins are
considered.
\subsection{Forcing Chains Strategy}
Looks at the consequences of filling the possibilities of a cell with two possible values.
If both possibilities lead to the same value elsewhere, that value must be entered. This is the most
complex strategy and can slow the solver down quite a bit.

\section{Results}
On the provided file \texttt{big5} (containing $10.000$ sudokus), the solver performs quite
well:
\begin{verbatim}
Puzzles: 10000
Solved: 9989 (99.9%)
Invalid puzzles: 0
Time: 72.26s (7.23ms average per puzzle)
\end{verbatim}

On harder puzzles (such as the provided \texttt{top100}) it is a lot slower:
\begin{verbatim}
Puzzles: 100
Solved: 99 (99.0%)
Invalid puzzles: 0
Time: 51.91s (519.10ms average per puzzle)
\end{verbatim}

Although this seems to be caused mainly by a few puzzles, that needs a lot of the
more complicated strategies to solve. Without these puzzles the performance is much
better:
\begin{verbatim}
Puzzles: 98
Solved: 97 (99.0%)
Invalid puzzles: 0
Time: 16.69s (170.27ms average per puzzle)
\end{verbatim}

Overall the solver can solve most puzzles (about 99\% of the harder ones) and is pretty
fast.

\end{document}

